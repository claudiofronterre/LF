\documentclass[11pt,]{article}
\usepackage[left=1in,top=1in,right=1in,bottom=1in]{geometry}
\newcommand*{\authorfont}{\fontfamily{phv}\selectfont}
\usepackage[]{mathpazo}
\usepackage{amsmath}
\usepackage{graphicx}
\usepackage{float}
  \usepackage[T1]{fontenc}
  \usepackage[utf8]{inputenc}



\usepackage{abstract}
\renewcommand{\abstractname}{}    % clear the title
\renewcommand{\absnamepos}{empty} % originally center

\renewenvironment{abstract}
 {{%
    \setlength{\leftmargin}{0mm}
    \setlength{\rightmargin}{\leftmargin}%
  }%
  \relax}
 {\endlist}

\makeatletter
\def\@maketitle{%
  \newpage
%  \null
%  \vskip 2em%
%  \begin{center}%
  \let \footnote \thanks
    {\fontsize{18}{20}\selectfont\raggedright  \setlength{\parindent}{0pt} \@title \par}%
}
%\fi
\makeatother



% Control depth of chapter/section numbering
\setcounter{secnumdepth}{3}






\title{Semi-mechanistic model and its implementation in R \thanks{\textbf{Corresponding author}:
\href{mailto:claudio.fronterre@lshtm.ac.uk}{\nolinkurl{claudio.fronterre@lshtm.ac.uk}}.}  }





\author{\Large Claudio Fronterrè\vspace{0.05in} \newline\normalsize\emph{London School of Hygiene and Tropical Medicine}  }


\date{}

\usepackage{titlesec}

\titleformat*{\section}{\normalsize\bfseries}
\titleformat*{\subsection}{\normalsize\bfseries}
\titleformat*{\subsubsection}{\normalsize\itshape}
\titleformat*{\paragraph}{\normalsize\itshape}
\titleformat*{\subparagraph}{\normalsize\itshape}


\usepackage{natbib}
\bibliographystyle{apalike}
\usepackage[strings]{underscore} % protect underscores in most circumstances



\newtheorem{hypothesis}{Hypothesis}
\usepackage{setspace}

\makeatletter
\@ifpackageloaded{hyperref}{}{%
\ifxetex
  \PassOptionsToPackage{hyphens}{url}\usepackage[setpagesize=false, % page size defined by xetex
              unicode=false, % unicode breaks when used with xetex
              xetex]{hyperref}
\else
  \PassOptionsToPackage{hyphens}{url}\usepackage[unicode=true]{hyperref}
\fi
}

\@ifpackageloaded{color}{
    \PassOptionsToPackage{usenames,dvipsnames}{color}
}{%
    \usepackage[usenames,dvipsnames]{color}
}
\makeatother
\hypersetup{breaklinks=true,
            bookmarks=true,
            pdfauthor={Claudio Fronterrè (London School of Hygiene and Tropical Medicine)},
             pdfkeywords = {lymphatic filariasis, antigeamia, geostatistics},  
            pdftitle={Semi-mechanistic model and its implementation in R},
            colorlinks=true,
            citecolor=cyan,
            urlcolor=blue,
            linkcolor=magenta,
            pdfborder={0 0 0}}
\urlstyle{same}  % don't use monospace font for urls

% set default figure placement to htbp
\makeatletter
\def\fps@figure{htbp}
\makeatother



% add tightlist ----------
\providecommand{\tightlist}{%
\setlength{\itemsep}{3pt}\setlength{\parskip}{0pt}}

\begin{document}
	
% \pagenumbering{arabic}% resets `page` counter to 1 
%
% \maketitle

{% \usefont{T1}{pnc}{m}{n}
\setlength{\parindent}{0pt}
\thispagestyle{plain}
{\fontsize{18}{20}\selectfont\raggedright 
\maketitle  % title \par  

}

{
   \vskip 13.5pt\relax \normalsize\fontsize{11}{12} 
\textbf{\authorfont Claudio Fronterrè} \hskip 15pt \emph{\small London School of Hygiene and Tropical Medicine}   

}

}








\begin{abstract}

    \hbox{\vrule height .2pt width 39.14pc}

    \vskip 8.5pt % \small 

\noindent This document provides details for the semi-mechanistic model developed
to estimate MF prevalence when two different diagnostic tools are used.
It also shows how to implement it in R.


\vskip 8.5pt \noindent \emph{Keywords}: lymphatic filariasis, antigeamia, geostatistics \par

    \hbox{\vrule height .2pt width 39.14pc}



\end{abstract}


\vskip 6.5pt


\noindent  \section{Introduction}\label{introduction}

Lymphatic filariasis (LF) is a mosquito-borne neglected tropical disease
targeted for global elimination by 2020. The majority of global cases
are caused by three species of nematode worms: \emph{Wuchereria
bancrofti}, \emph{Brugia malayi} and \emph{Brugia timori}. These
filariae parasites are transmitted by various species of mosquito
vectors from the genera Anopheles, Aedes, Culex, Mansonia and
Ochlerotatus. In recent years, the mapping of LF has been greatly
facilitated by the use of simple and rapid detection tests for \emph{W.
bancrofti} (antigen-based test) and \emph{Brugia} (antibody-based test),
based on the immuno-chromatographic test (ICT card test), which avoids
the need to collect blood at night and the time-consuming preparation
and examination of blood slides. While it is known that estimates of
antigenaemia are generally higher than estimates of microfilaraemia
(MF), the extent and spatial heterogeneity of this relationship is not
clear. Even if the scientific output of interest is the prevalence of
microfilaraemia, the small number of mapping surveys that measure mf is
low and it is decreasing due to the diffusion and cost-effectivness of
ICT tests. Our goal is to use the abundace of ICT prevalence surveys and
the relationship between ICT and MF prevalence to predict
microfilaraemia prevalence at unobserved locations.

\section{Modeling framework}\label{modeling-framework}

We can define our set of data as follow:
\[\mathcal{D}=\left\{ \left(x_{i,j},n_{i,j},y_{i,j}\right):x_{i,j}\in A\right\} ,i=1,\ldots,n_{j},j=1,2; \label{eq:data}\]
where \(x\) is the geograpich location of the mapping survey, \(y\) is
the number of people infected out of \(n\) examined and \(j\) indicate
which type of prevalence was measured, \(1 = MF\) and \(2 = ICT\). Our
final target of estimation is the predictive distribution of MF
prevalance given ICT prevalence \(\left[Y_1 \mid Y_2, S\right]\). We can
consider the observed prevalence as the realisation of a binomial random
variable
\[Y_{i,j} \mid S_j(x_ij),Z_{i,j} \sim \text{Binomial}(n_i,j,p_j(x_{i,j})).\]
We use biological information to define the probability of being tested
as postive \[
p_{j}\left(x\right)=\begin{cases}
1-\exp\left\{ -\lambda\left(x\right)\left[1-\exp\left(-\alpha\right)\right]\right\} & j=1 \\
\phi\left\{ 1-\exp\left[-\lambda\left(x\right)\right]\right\} & j=2
\end{cases}
\] where \(\alpha\) is a parameter that control the reproductive rate of
MF, \(\phi\) is the sensitivy of the ICT test and \(\lambda(x)\) is the
mean number of adult worms in a sampled individual. The last one is
assumed to vary spatially using the following specification
\[\log{\lambda(x)}=d(x)^\text{T}\beta+S(x)+Z.\]

\subsection{Lieklihood}\label{lieklihood}

Let \(\theta\) be the vector of model parameters to be estimated. The
log-likelihood for this model is
\[l\left(\theta\right)=\sum_{j=1}^{2}\left[n_{j}\sum_{i=1}^{n}\log\left(1-p_{i,j}\right)+\sum_{i=1}^{n}\log\left(\frac{p_{i,j}}{1-p_{i,j}}\right)\right].\]

\section{Implementation in R}\label{implementation-in-r}




\newpage
\singlespacing 
\bibliography{biblio2.bib}

\end{document}

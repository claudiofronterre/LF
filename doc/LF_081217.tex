\documentclass[12pt,a4paper]{article}
\usepackage[utf8]{inputenc}
\usepackage{amsmath, amsthm, empheq,mathtools}
%\usepackage{amsmath,amstext,amsgen,amsbsy,amsopn,amsfonts,graphicx,
%overcite,theorem}
\usepackage{listings}
\usepackage{xcolor}
\usepackage{tcolorbox}
\usepackage{titlesec}
\usepackage{fixmath}    %for mathbold
\titleformat*{\section}{\Large\bfseries}
\titleformat*{\subsection}{\normalsize\bfseries}
\usepackage{amsfonts}
\renewcommand{\baselinestretch}{1.5}
%\linespread{1.3}
\usepackage{amssymb}
\usepackage[authoryear]{natbib} %for reference
\usepackage{graphicx}   %for subplots
\usepackage{subfig} %for plots
%\setcitestyle{authoryear}
%\setcitestyle{authoryear, open={((},close={))}
\usepackage{moreverb}
\usepackage{parskip}
\usepackage{url}
%\usepackage{graphicx}
\usepackage{lmodern}
\usepackage[flushleft]{threeparttable} 

%for Algorithm

\usepackage{algorithmic}
\usepackage{algorithm}

%ref Capital at start
%\usepackage{cleveref} %\Cref for capital letter


%the end of Algorithm


\usepackage{hyperref}
\usepackage{psfrag}
%\usepackage{kpfonts}
\usepackage{float}
\usepackage{color}
\usepackage[left=2cm,right=2cm,top=2cm,bottom=2cm]{geometry}
%\rhead{\includegraphics[scale=0.2]{logo.eps}}
%\cfoot{\includegraphics[scale=0.2]{logo.eps}}


\thispagestyle{empty}


\definecolor{mygreen}{rgb}{0,0.6,0}
\definecolor{mygray}{rgb}{0.5,0.5,0.5}
\definecolor{mymauve}{rgb}{0.58,0,0.82}
\newtheorem{theorem}{Theorem}[section]
\newtheorem{corollary}{Corollary}[theorem]
\newtheorem{lemma}[theorem]{Lemma}


%To reference subfigure
\captionsetup[subfigure]{subrefformat=simple,labelformat=simple,listofformat=subsimple}
\renewcommand\thesubfigure{(\alph{subfigure})}


\lstset{ %
  backgroundcolor=\color{white},   % choose the background color; you must add \usepackage{color} or \usepackage{xcolor}
  basicstyle=\footnotesize,        % the size of the fonts that are used for the code
  breakatwhitespace=false,         % sets if automatic breaks should only happen at whitespace
  breaklines=true,                 % sets automatic line breaking
  captionpos=b,                    % sets the caption-position to bottom
  commentstyle=\color{mygreen},    % comment style
  deletekeywords={...},            % if you want to delete keywords from the given language
  escapeinside={\%*}{*)},          % if you want to add LaTeX within your code
  extendedchars=true,              % lets you use non-ASCII characters; for 8-bits encodings only, does not work with UTF-8
  frame=single,	                   % adds a frame around the code
  keepspaces=true,                 % keeps spaces in text, useful for keeping indentation of code (possibly needs columns=flexible)
  keywordstyle=\color{blue},       % keyword style
  language=Octave,                 % the language of the code
  otherkeywords={*,...},           % if you want to add more keywords to the set
  numbers=left,                    % where to put the line-numbers; possible values are (none, left, right)
  numbersep=5pt,                   % how far the line-numbers are from the code
  numberstyle=\tiny\color{mygray}, % the style that is used for the line-numbers
  rulecolor=\color{black},         % if not set, the frame-color may be changed on line-breaks within not-black text (e.g. comments (green here))
  showspaces=false,                % show spaces everywhere adding particular underscores; it overrides 'showstringspaces'
  showstringspaces=false,          % underline spaces within strings only
  showtabs=false,                  % show tabs within strings adding particular underscores
  stepnumber=2,                    % the step between two line-numbers. If it's 1, each line will be numbered
  stringstyle=\color{mymauve},     % string literal style
  tabsize=2,	                   % sets default tabsize to 2 spaces
  title=\lstname                   % show the filename of files included with \lstinputlisting; also try caption instead of title
}
\newcommand{\quotes}[1]{``#1''}
%\title{Continuously indexed Gaussian fields for Aggregated Data}
\title{Geostatistical modelling of the relationship between microfilariae and antigenaemia prevalence of lymphatic filariasis infections}
\author{Claudio Fronterr\`e$^1$, Jorge Cano$^1$, Rachel Pullan$^1$,  Emanuele Giorgi$^2$}
\begin{document}
\maketitle
\begin{abstract}
...
\end{abstract}
\textbf{Keywords:} ...

\section{Background}
\label{sec:background}
\begin{itemize}
\item Give background information in LF.

For Jorge/Rachel to fill.

\item Define the diagnostic procedures: microfilariae counts and immunochromatographic test (ICT). Contrast advantages and disadvantages of both diagnostics. 

To test 

\item Define objective of the study: development of a geostatisitcal model for the diagnostics procedures. Briefly outline advantages of this over standard methods. 
\item Describe outline of the paper.
\end{itemize}

\section{Bivariate geo-statistical modelling of microfilariae and antigenaemia prevalence}
\label{sec:biv_model}
\begin{itemize}
\item Define the standard format of the data and define the problem in statistical language. 
\end{itemize}

\subsection{Semi-mechanistic modelling}
\label{subsec:semi_mech}
\begin{itemize}
\item Summarize Irvine et al. (2016). 
\item Discuss density-dependence.
\item Extend the Irvine model to a geostatistical model, with and without density-dependence.
\end{itemize}

\subsection{Empirical modelling}
\label{subsec:empirical}
\begin{itemize}
\item Describe existing geostatistical methods for bivariate modelling; e.g. Crainiceanu and Diggle (2008). Cite Benjamin's paper which is going to be submitted by the end of January, where the empirical approach is developed. 
\item Formulate a model for LF mapping that uses the two diagnostics. 
\end{itemize}

\section{Inference}
\label{sec:inference}
\begin{itemize}
\item Define the likelihood function; you can do this by using a general notation that is valid for both the empirical and semi-mechanistic approach. 
\item Outline the Monte Carlo maximum likelihood approach.
\end{itemize}

\section{Simulation study}
\label{sec:simulation}
\begin{itemize}
\item Define objective of the simulation study: understanding the impact of model misspecification on predictive inferences. 
\item Define scenarios (to discuss).
\item Report and comment results.
\end{itemize}

\section{Application: lymphatic filariasis mapping in West Africa}
\label{sec:application}
\begin{itemize}
\item Description of the data.
\item Exploratory analysis (to discuss).
\item Fitting of the two models. Report estimates and comment those. 
\item Predictions and comparison.
\end{itemize}

\section{Discussion}
\label{sec:discussion}
\begin{itemize}
\item Summarize findings. 
\item Answer the question: what model should we use?
\item Comment on limitations of the analysis.
\item Can the work be extended? Can the methodology be applied to other diseases?
\end{itemize}

\bibliographystyle{biometrika.bst}
\bibliography{biblio.bib}
\end{document}